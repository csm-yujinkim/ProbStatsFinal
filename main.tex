\documentclass[letterpaper]{article}

\usepackage[utf8]{inputenc}
\usepackage[T1]{fontenc}
\usepackage{amsmath}
\usepackage{booktabs}
\usepackage{graphicx}

\begin{document}
    \section{Digital Signature}
    Signed Name:
    YuJin Kim

    Printed Name:
    YuJin Kim

    Section:
    B

    \section{Answers}
    \begin{enumerate}
        \item
        There are 15 items.
        \begin{enumerate}
            \item               % 1-a
            \begin{enumerate}
                \item
                1st quartile: Use index $(.25)(16)=4$.
                The fourth item is $1.6$.
                \item
                2nd quartile: The eighth item is $2.3$.
                \item
                3rd quartile: The twelfth item is $3.6$.
            \end{enumerate}
            \item               % 1-b
            The median (2nd quartile) is $2.3$, which is less than the mean.
            Therefore, the distribution shows skew to the left.
        \end{enumerate}
        \item
        \begin{align*}
            \int_{1}^{\infty} 2e^{-2x} dx
            &= \lim_{b\to\infty}\int_{1}^{b} 2e^{-2x} dx \\
            &= -\lim_{b\to\infty}\left[e^{-2x}\right]_{1}^{b} \\
            &= -\lim_{b\to\infty}\left[e^{-2b} - e^{-2}\right] \\
            &= -\lim_{b\to\infty}\left[0 - e^{-2}\right] \\
            &= \frac{1}{e^2} \\
            &\approx 0.13534
        \end{align*}
        \item
        \begin{enumerate}
            \item % Problem 3 to-do
        \end{enumerate}
        \item
        \begin{enumerate}
            \item
            \begin{align*}
                P(A\mid B) &= \frac{P(A\cap B)}{P(B)} \\
                P(B) P(A\mid B) &= P(A\cap B) \\
                0.2\times 0.3 &= 0.06 \\
                \\
                P(A\cup B)
                &= P(A) + P(B) - P(A\cap B) \\
                &= 0.4 + 0.2 - 0.06 \\
                &= 0.54
            \end{align*}
            \item
            $P(A\mid B) \ne P(A)$, so $A$ and $B$ are not independent.
            \item
            $P(A\cap B) > 0$, so they are not mutually exclusive.
        \end{enumerate}
        \item
        \begin{enumerate}
            \item
            \begin{itemize}
                \item
                Mean: $5 - 6 = -1$
                \item
                Variance: $3 + 1 = 4$
            \end{itemize}
            \item
            \begin{align*}
                z
                &= \frac{0.5 - (-1)}{\sqrt{4}} \\
                &= 0.75
            \end{align*}
            According to R, the area to the right side of $z$ (i.e. collection of points $> z$) in the normal curve is $0.23$.
            Thus, the probability is $0.23$.
        \end{enumerate}
        \item
        $H_0: \mu \le 4, H_1: \mu > 4$, where $\mu$ is life of battery in years.
        \begin{enumerate}
            \item
            Type I error: rejecting $H_0$ when it is true.
            \item
            No error.
            \item
            P-value is too high to reject $H_0$, so it is not rejected.
            $H_0$ is actually true, so it should not be rejected.
            No error occurs.
        \end{enumerate}
        \item
        P-value decreases as $z$ moves away from the mean.
        $|z|$ increases if the standard error decreases.
        The standard error decreases if the number of samples increases.
        So, \textbf{p-value decreases} as the number of samples increases.
        \item
        \begin{align*}
            p &= 0.25 \\
            \hat{p} &= 138/500 = 0.276\textrm{ (exact)} \\
            n &= 500 \\
            s &= \sqrt{\frac{p(1-p)}{n}} \approx 0.01936492 \\
            z &= \frac{\hat{p} - p}{s} \approx 1.342634
        \end{align*}
        According to R, the p-value is approximately $0.09$.
        Since the p-value is too high, it cannot be concluded that the proportion of smokers is higher than 25\%.
        \item
        \begin{enumerate}
            \item T % Probably, linear combinations of normal are also normal
            \item F ($\mu_X = 10.5$)
            \item F % *mutually exclusive*
            \item F % Not if true variance is known
            \item F % \mu is a parameter so probability doesn't apply
            \item F % positive correlation does not decide which sign of the slope it is
        \end{enumerate}
        \item
        \begin{enumerate}
            \item T % small sample, normally distributed, assumed no outliers, and true sd unknown
            \item
            \begin{align*}
                \bar{X} &= 2.7 \\
                s &= 3.43 \\
                n &= 10 \\
                \\
                e &= \frac{s}{\sqrt{n}} \approx 1.085 \\
                t &= \frac{\bar{X} - \mu}{e} \approx -0.2766 \\
                \mathrm{df} &= n - 1 = 9
            \end{align*}
            $\bar{X}$ : sample mean.
            $s$ : sample standard deviation.
            $n$ : sample size.
            $e$ : standard error of the mean.
            $t$ : $t$-statistic
            $\mathrm{df}$ : degrees of freedom.

            According to the $t$-table with 9 degrees of freedom, the \textbf{P value is greater than 0.100}.
            \item
            \begin{itemize}
                \item F % mean is said to be 1.001. 95% interval says mean is 1.988.
                \item T
                \item % to-do
                \item % to-do
            \end{itemize}
        \end{enumerate}
        \item
        \begin{enumerate}
            \item
            $(\textrm{\# of columns} - 1)\times(\textrm{\# of rows} - 1) = 2\times 3 = 6$
            \item
            See Table \ref{tab:richness}. The expected value is $25$.
            \begin{table}[h]
                \centering
                \resizebox{\textwidth}{!}{%
                \begin{tabular}{@{}l|lll|l@{}}
                \toprule
                                     & Marginally Rich  & Comfortably Rich & Super Rich       & TOTAL \\ \midrule
                No college           & 25               & 25               & 25               & 75    \\
                Some college         & 10               & 10               & 10               & 30    \\
                Undergraduate degree & 51.3333333333333 & 51.3333333333333 & 51.3333333333333 & 154   \\
                Postgraduate study   & 13.6666666666667 & 13.6666666666667 & 13.6666666666667 & 41    \\ \midrule
                TOTAL                & 100              & 100              & 100              & 300   \\ \bottomrule
                \end{tabular}%
                }
                \caption{Each cell is calculated by multiplying column and row totals and dividing it by the total.}
                \label{tab:richness}
            \end{table}
            \item
            I disagree.
            The \textit{expected values} matter, rather than the observed ones.
            In Table \ref{tab:richness}, all expected values are above $5$, so the $\chi^2$-test will be valid.
            \item
            According to the $\chi^2$-table, with 6 degrees of freedom, $\chi^2 = 14.45$ corresponds to a p-value of $0.025$.
            This p-value is low enough (less than the level $0.05$) to assert the claim.
        \end{enumerate}
        \item
        \begin{enumerate}
            \item
            Goodness of fit.
            % Checking distribution
            \item
            The average value $25$.
            \item
            According to the $\chi^2$-table, the p-value will be between $0.1$ and $0.9$.
            So, it will be too high to conclude that the proportion of students is not equal.
        \end{enumerate}
        \item \begin{enumerate}
            \item
            above average
            \item
            $-0.95$
        \end{enumerate}
    \end{enumerate}
\end{document}
