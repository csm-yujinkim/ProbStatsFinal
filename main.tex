\documentclass[letterpaper]{article}

\usepackage[utf8]{inputenc}
\usepackage[T1]{fontenc}
\usepackage{amsmath}

\begin{document}
    \section{Digital Signature}
    Signed Name:
    YuJin Kim

    Printed Name:
    YuJin Kim

    Section:

    \section{Answers}
    \begin{enumerate}
        \item
        There are 15 items.
        \begin{enumerate}
            \item               % 1-a
            \begin{enumerate}
                \item
                1st quartile: Use index $(.25)(16)=4$.
                The fourth item is $1.6$.
                \item
                2nd quartile: The eighth item is $2.3$.
                \item
                3rd quartile: The twelfth item is $3.6$.
            \end{enumerate}
            \item               % 1-b
            The median (2nd quartile) is $2.3$, which is less than the mean.
            Therefore, the distribution shows skew to the left.
        \end{enumerate}
        \item
        \begin{align*}
            \int_{1}^{\infty} 2e^{-2x} dx
            &= \lim_{b\to\infty}\int_{1}^{b} 2e^{-2x} dx \\
            &= -\lim_{b\to\infty}\left[e^{-2x}\right]_{1}^{b} \\
            &= -\lim_{b\to\infty}\left[e^{-2b} - e^{-2}\right] \\
            &= -\lim_{b\to\infty}\left[0 - e^{-2}\right] \\
            &= \frac{1}{e^2} \\
            &\approx 0.13534
        \end{align*}
        \item
        \begin{enumerate}
            \item % Problem 3 to-do
        \end{enumerate}
        \item
        \begin{enumerate}
            \item
            \begin{align*}
                P(A\mid B) &= \frac{P(A\cap B)}{P(B)} \\
                P(B) P(A\mid B) &= P(A\cap B) \\
                0.2\times 0.3 &= 0.06 \\
                \\
                P(A\cup B)
                &= P(A) + P(B) - P(A\cap B) \\
                &= 0.4 + 0.2 - 0.06 \\
                &= 0.54
            \end{align*}
            \item
            $P(A\mid B) \ne P(A)$, so $A$ and $B$ are not independent.
            \item
            $P(A\cap B) > 0$, so they are not mutually exclusive.
        \end{enumerate}
        \item
        \begin{enumerate}
            \item
            \begin{itemize}
                \item
                Mean: $5 - 6 = -1$
                \item
                Variance: $3 + 1 = 4$
            \end{itemize}
            \item
            \begin{align*}
                z
                &= \frac{0.5 - (-1)}{\sqrt{4}} \\
                &= 0.75
            \end{align*}
            According to R, the area to the right side of $z$ (i.e. collection of points $> z$) in the normal curve is $0.23$.
            Thus, the probability is $0.23$.
        \end{enumerate}
        \item
        $H_0: \mu \le 4, H_1: \mu > 4$, where $\mu$ is life of battery in years.
        \begin{enumerate}
            \item
            Type I error: rejecting $H_0$ when it is true.
            \item
            No error.
            \item
            P-value is too high to reject $H_0$, so it is not rejected.
            $H_0$ is actually true, so it should not be rejected.
            No error occurs.
        \end{enumerate}
        \item
        P-value decreases as $z$ moves away from the mean.
        $|z|$ increases if the standard error decreases.
        The standard error decreases if the number of samples increases.
        So, \textbf{p-value decreases} as the number of samples increases.
        \item
        \begin{align*}
            p &= 0.25 \\
            \hat{p} &= 138/500 = 0.276\textrm{ (exact)} \\
            n &= 500 \\
            s &= \sqrt{\frac{p(1-p)}{n}} \approx 0.01936492 \\
            z &= \frac{\hat{p} - p}{s} \approx 1.342634
        \end{align*}
        According to R, the p-value is approximately $0.09$.
        Since the p-value is too high, it cannot be concluded that the proportion of smokers is higher than 25\%.
    \end{enumerate}
\end{document}
